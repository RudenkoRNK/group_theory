\newpage
\section{Справка}
\subsection{Перестановки}
\begin{definition}
  \emph{Циклом} называется перестановка, которая часть (неупорядоченных) элементов переставляет циклически, а остальные оставляет на месте.
\end{definition}

\begin{definition}
  Циклы называются \emph{независимыми} если они не содержат общих элементов, на которые они нетривиально действуют.
\end{definition}

\begin{theorem}
  Любую перестановку можно представить в виде композиции независимых циклов.
\end{theorem}

\begin{definition}
  Цикл называется \emph{транспозицией}, если он переставляет всего два элемента.
\end{definition}

\begin{definition}
  \emph{Четностью перестановки} $\sgn\sigma$ называется четность числа беспорядков $N(\sigma)$:\\
  \[N(\sigma)\equiv \left|\left\{\left(i,j\right)\colon i<j \land \sigma(i)>\sigma(j)\right\}\right|\]
\end{definition}

\begin{theorem}\
  \begin{enumerate}
    \item Четность числа транспозиций в которую раскладывается перестановка является инвариантом перестановки и равна четности перестановки.
    \item $\sgn\sigma\tau=\sgn\sigma\sgn\tau$
  \end{enumerate}
\end{theorem}
