\documentclass[a5paper,12pt]{article}

%\usepackage{extsizes}                  % Возможность сделать 14-й шрифт

%%% Программирование
\usepackage{etoolbox}                  % Логические операторы
\usepackage{iftex}                     % Определение компилятора

%%% Гиперссылки
\usepackage{hyperref}
\usepackage[usenames,dvipsnames,svgnames,table,rgb]{xcolor}
\hypersetup{
    unicode=true,                      % Русские буквы в раздела PDF
    pdftitle={Заголовок},              % Заголовок
    pdfauthor={Автор},                 % Автор
    pdfsubject={Тема},                 % Тема
    pdfcreator={Создатель},            % Создатель
    pdfproducer={Производитель},       % Производитель
    pdfkeywords={keyword1} {keyword2}, % Ключевые слова
    colorlinks=true,                   % false: ссылки в рамках; true: цветные ссылки
    linkcolor=red,                     % Внутренние ссылки
    citecolor=green,                   % На библиографию
    filecolor=magenta,                 % На файлы
    urlcolor=blue                      % На URL
}

%%% Работа с русским языком
\usepackage[english,russian]{babel}        % Пакет многоязыковой вёрстки
\ifXeTeX
    % У вас должны быть установлены CMU шрифты!
    \usepackage{fontspec}                      % Загрузка шрифтов Open Type, True Type и др.
    \defaultfontfeatures{Ligatures={TeX},      %
        Renderer=Basic}                        % Свойства шрифтов по умолчанию
    \setmainfont[Ligatures={TeX,Historic}]     %
    {CMU Serif}                                % Основной шрифт документа
    \setsansfont{Times New Roman}              % Шрифт без засечек
    \setmonofont{CMU Typewriter Text}          % Моноширинный шрифт}
\else
    \RequirePDFTeX
    \usepackage{cmap}                      % Поиск в PDF
    \usepackage{mathtext}                  % Русские буквы в формулах
    \usepackage[T2A]{fontenc}              % Кодировка
    \usepackage[utf8]{inputenc}            % Кодировка исходного текста
\fi
\usepackage{csquotes}
\usepackage{indentfirst}               % Красная строка для первого абзаца
\usepackage{parskip}
\setlength{\parindent}{0ex}            % Убрать красную строку
\frenchspacing

%%% Шрифты
\usepackage{euscript}                        % Шрифт Евклид
\usepackage{mathrsfs}                        % Красивый матшрифт
\usepackage{textcomp}                        % Symbols like copyright, degree
\usepackage{soul}                            % Модификаторы начертания
%\renewcommand{\familydefault}{\sfdefault}   % Шрифт без засечек

%%% Страница
\usepackage{multicol}                  % Несколько колонок
\usepackage{geometry}                  % Простой способ задавать поля
\geometry{top=20mm}
\geometry{bottom=20mm}
\geometry{left=10mm}
\geometry{right=10mm}

%%% Колонтитулы
\usepackage{lastpage}                  % Число страниц в документе
\usepackage{fancyhdr}
\pagestyle{fancy}
\renewcommand{\headrulewidth}{0mm}     % Толщина линейки, отчеркивающей верхний колонтитул
\lfoot{}
\rfoot{}
\rhead{}
\chead{}
\lhead{}
%\cfoot{Нижний в центре}               % По умолчанию здесь номер страницы

%%% Интерлиньяж
\usepackage{setspace}
%\singlespacing                        % Интерлиньяж 1
%\onehalfspacing                       % Интерлиньяж 1.5
%\doublespacing                        % Интерлиньяж 2

%%% Работа с картинками
\usepackage{graphicx}                  % Для вставки рисунков
\usepackage{wrapfig}                   % Обтекание рисунков и таблиц текстом
\usepackage{epstopdf}                  % Конвертер
\graphicspath{{images/}}               % Папки с картинками
\setlength\fboxsep{3pt}                % Отступ рамки \fbox{} от рисунка
\setlength\fboxrule{1pt}               % Толщина линий рамки \fbox{}

%%% Работа с таблицами
\usepackage{array,booktabs}            % Дополнительная работа с таблицами
\usepackage{tabularx,tabulary}         % Таблицы, выровененные по ширине или высоте
\usepackage{longtable}                 % Длинные таблицы
\usepackage{multirow}                  % Слияние строк в таблице

%%% Работа с графикой
\usepackage{tikz}
\usetikzlibrary{matrix}
\usepackage{pgfplots}
\pgfplotsset{compat=1.16}
\usepackage{pgfplotstable}

%\mathtoolsset{showonlyrefs=true}      % Показывать номера только у формул со сслыкой \eqref{} в тексте.
%\usepackage{leqno}                    % Нумерация формул слева

%%% Прочее
\usepackage{enumitem}
%\usepackage{todonotes}
\usepackage{science/science}
