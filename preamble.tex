\documentclass[a4paper,12pt]{article}
% 13pt size font
\usepackage[fontsize=12pt]{scrextend}

%=============================== Packages ===============================

% A comprehensive SI units package, keep it at the top
\usepackage{siunitx}

% Encodings
\usepackage[T1,T2A]{fontenc}
\usepackage{fontspec}
\usepackage[english,russian]{babel}


% NewDocumentCommand and conditional expressions
\usepackage{xparse}
\usepackage{iftex}
\usepackage{etoolbox}

\usepackage[usenames,dvipsnames,svgnames,table,rgb]{xcolor}

\usepackage{enumitem}

\usepackage{multicol}

% Margins
\usepackage{geometry}

% Centered sections
\usepackage{sectsty}

% More intelligent hyphenation
\usepackage{microtype}

% Tables
\usepackage{array,booktabs}
\usepackage{tabularx,tabulary}
\usepackage{longtable}
\usepackage{multirow}

% Pictures
\usepackage{graphicx}
\usepackage{wrapfig}
\usepackage{epstopdf}
\usepackage{adjustbox}

% Bibliography
\usepackage[backend=biber,
    bibencoding=utf8,
    sorting=nty,
    maxcitenames=2,
    style=numeric-comp]
{biblatex}

% Search in PDF document
\usepackage{cmap}

\usepackage{hyperref}
\usepackage[font=small,figurename=Рисунок]{caption}

% Correct quotation
\usepackage{fvextra}
\usepackage{csquotes}

% First line indent
\usepackage{indentfirst}
\usepackage{parskip}

% Symbols like copyright, degree
\usepackage{textcomp}

% Bold, italics, etc.
\usepackage{soul}

% Number of pages in document
\usepackage{lastpage}

% Local style setup
\usepackage{afterpage}

% Header and footer
\usepackage{fancyhdr}

% Linespacing
\usepackage{setspace}

% Russian letters in formulas
\usepackage{mathtext}

\usepackage{amsmath}

% Extended math symbol collection
\usepackage{amssymb}

\usepackage{amsfonts}

\usepackage{amsthm}

% Fix bugs in ams
\usepackage{mathtools}

% Math font with \mathscr command
\usepackage{mathrsfs}

% Additional math symbols
\usepackage{stmaryrd}

% Code highlighting
\usepackage[outputdir=build,newfloat]{minted}

% \nicefrac \unitfrac and \unit command
\usepackage{units}

% \qty, \norm, \abs, \div, \grad, \curl, \laplacian, etc.
\usepackage{physics}

% Graphics
\usepackage{tikz}
\usetikzlibrary{matrix}
\usepackage{pgfplots}
\pgfplotsset{compat=1.16}
\usepackage{pgfplotstable}


%=============================== Customizations ===============================
\defaultfontfeatures{Renderer=Basic,Ligatures={TeX}}
\setmainfont{CMU Serif}
\setsansfont{CMU Sans Serif}
\setmonofont{CMU Typewriter Text}

% Set linespacing
%\singlespacing
%\onehalfspacing
%\doublespacing

% Set up margins
\geometry{top=20mm}
\geometry{bottom=20mm}
\geometry{left=10mm}
\geometry{right=10mm}

% Set up headers
\pagestyle{fancy}
\renewcommand{\headrulewidth}{0mm}
\lfoot{}
\rfoot{}
\rhead{}
\chead{}
\lhead{}
%\cfoot{Нижний в центре}

\graphicspath{{images/}{build/images/}{build/}}

% Picture box margins
\setlength\fboxsep{3pt}
\setlength\fboxrule{1pt}

\newcolumntype{x}[1]{>{\centering\let\newline\\\arraybackslash\hspace{0pt}}p{#1}}

% First line indent
\setlength{\parindent}{0cm}
\frenchspacing

\hypersetup{
    unicode=true,
    pdftitle={Конспект по теории групп},
    pdfauthor={Никита Руденко},
    % False means refs in boxes, true means color refs
    colorlinks=true,
    linkcolor=red,
    citecolor=blue,
    filecolor=magenta,
    urlcolor=blue
}

\captionsetup[figure]{
    labelformat=simple,
    labelsep=endash,
    justification=centering
}

\captionsetup[listing]{
    labelformat=simple,
    labelsep=endash,
    justification=centering
}

\SetupFloatingEnvironment{listing}{name=Листинг}

% Center sections
%\allsectionsfont{\centering}

% Only show numbers if formula is referenced
%\mathtoolsset{showonlyrefs=true}

% Display formula number on the left
%\usepackage{leqno}


\addbibresource{bibliography.bib}

%=============================== Support commands ===============================
\makeatletter
\NewDocumentCommand{\swapcommands}{m m}{%
    \let\@swapcommands #1%
    \let #1 #2%
    \let #2\@swapcommands%
}
\NewDocumentCommand{\wrapensuremath}{m}{%
    % Throw error, if "\N#1" is already defined.
    \expandafter\@ifdefinable\csname #1@ensuremath\endcsname{%
        % Save old meaning
        \expandafter
        \let\csname #1@ensuremath\expandafter\endcsname
        \csname #1\endcsname
        % Define new macro
        \expandafter\edef\csname #1\endcsname{%
            \noexpand\ensuremath{%
                \expandafter\noexpand\csname #1@ensuremath\endcsname
            }%
        }%
    }%
}

% Parse curly or round brackets
\NewDocumentCommand{\@PCRB}{m m}{
    \IfNoValueTF{#1}{\IfNoValueTF{#2}{}{\quantity(#2)}}
    {#1 \IfNoValueTF{#2}{}{(#2)}}
}
% Parse math operator
\NewDocumentCommand{\parsemathoperator}{m m m}{
    \ensuremath{#1\@PCRB{#2}{#3}}
}




%=============================== Russian typography ===============================
\NewDocumentCommand{\varle}{}{\leqslant}
\NewDocumentCommand{\varge}{}{\geqslant}
\swapcommands{\epsilon}{\varepsilon}
\swapcommands{\phi}{\varphi}
\swapcommands{\kappa}{\varkappa}
\swapcommands{\emptyset}{\varnothing}
\swapcommands{\le}{\varle}
\swapcommands{\ge}{\varge}

%=============================== Ensure math for common symbols ===============================
\wrapensuremath{alpha}
\wrapensuremath{beta}
\wrapensuremath{gamma}   %\wrapensuremath{Gamma}
\wrapensuremath{delta}   %\wrapensuremath{Delta}
\wrapensuremath{epsilon}
\wrapensuremath{zeta}
\wrapensuremath{eta}
\wrapensuremath{theta}   %\wrapensuremath{Theta}
\wrapensuremath{iota}
\wrapensuremath{kappa}
\wrapensuremath{lambda}  %\wrapensuremath{Lambda}
\wrapensuremath{mu}
\wrapensuremath{nu}
\wrapensuremath{xi}      %\wrapensuremath{Xi}
\wrapensuremath{pi}      %\wrapensuremath{Pi}
\wrapensuremath{rho}
\wrapensuremath{sigma}   %\wrapensuremath{Sigma}
\wrapensuremath{tau}
\wrapensuremath{upsilon} %\wrapensuremath{Upsilon}
\wrapensuremath{phi}     %\wrapensuremath{Phi}
\wrapensuremath{chi}
\wrapensuremath{psi}     %\wrapensuremath{Psi}
\wrapensuremath{omega}   %\wrapensuremath{Omega}
\wrapensuremath{iff}
\wrapensuremath{rightarrow}
\wrapensuremath{leftarrow}
\wrapensuremath{implies}
\wrapensuremath{impliedby}

%=============================== Common math ===============================
\NewDocumentCommand{\aqty}{m}{\ensuremath{\left<#1\right>}}
\NewDocumentCommand{\n}{m}{\ensuremath{\centernot{#1}}}
\NewDocumentCommand{\x}{}{\ensuremath{\times}}
\NewDocumentCommand{\ox}{}{\ensuremath{\otimes}}
\NewDocumentCommand{\transpose}{m}{#1^\top}
\NewDocumentCommand{\T}{}{^\top}
\NewDocumentCommand{\orthogonal}{}{^\top}
\NewDocumentCommand{\ort}{}{^\top}
\NewDocumentCommand{\herm}{}{^\dagger}
\NewDocumentCommand{\suchthat}{}{\colon}
\NewDocumentCommand{\func}{m m m}{\ensuremath{#1\colon #2\rightarrow #3}} % Function: X -> Y
\NewDocumentCommand{\isomorph}{}{\ensuremath{\cong}}
\NewDocumentCommand{\divby}{}{\mathrel{\rotatebox{90}{\ensuremath{\hskip-1pt.{}.{}.}}}}
\NewDocumentCommand{\ord}{g d()}{\parsemathoperator{\operatorname{ord}}{#1}{#2}}
\NewDocumentCommand{\sgn}{g d()}{\parsemathoperator{\operatorname{sgn}}{#1}{#2}}
\NewDocumentCommand{\Id }{g d()}{\parsemathoperator{\operatorname{Id}}{#1}{#2}}

%=============================== Sets ===============================
\NewDocumentCommand{\set}{m m}{\ensuremath{\qty{#1\colon #2}}}
\NewDocumentCommand{\powerset}{g d()}{\parsemathoperator{\mathcal{P}}{#1}{#2}}
\NewDocumentCommand{\permset}{g d()}{\parsemathoperator{\mathcal{S}}{#1}{#2}}
\NewDocumentCommand{\N}{}{\ensuremath{\mathbb{N}}}
\NewDocumentCommand{\Z}{}{\ensuremath{\mathbb{Z}}}
\NewDocumentCommand{\Q}{}{\ensuremath{\mathbb{Q}}}
\NewDocumentCommand{\R}{}{\ensuremath{\mathbb{R}}}
\RenewDocumentCommand{\C}{}{\ensuremath{\mathbb{C}}} % was U+030F "COMBINING DOUBLE GRAVE ACCENT".
\NewDocumentCommand{\union}{}{\ensuremath{\cup}}
\NewDocumentCommand{\intxn}{}{\ensuremath{\cap}}
\NewDocumentCommand{\Union}{}{\ensuremath{\bigcup}}
\NewDocumentCommand{\Intxn}{}{\ensuremath{\bigcap}}

%=============================== Mathematical logic ===============================
\NewDocumentCommand{\imply}{}{\ensuremath{\:\rightarrow\:}}
\NewDocumentCommand{\implied}{}{\ensuremath{\:\leftarrow\:}}
\NewDocumentCommand{\eqdef}{}{\ensuremath{\:\longleftrightarrow\:}}
\swapcommands{\equiv}{\eqdef}
\NewDocumentCommand{\iffdef}{}{\ensuremath{\aqty{\eqdef}}} % iff by definition
\NewDocumentCommand{\eqsym}{}{\ensuremath{\eqcirc}} % Equivalent symbol by symbol
\NewDocumentCommand{\A}{m g d()}{\parsemathoperator{\forall #1\:}{#2}{#3}}
\NewDocumentCommand{\E}{m g d()}{\parsemathoperator{\exists #1\:}{#2}{#3}}
\RenewDocumentCommand{\a}{}{\ensuremath{\land}}
\RenewDocumentCommand{\o}{}{\ensuremath{\lor}}   % was little emptyset


%=============================== Group theory ===============================
\RenewDocumentCommand{\ker}{g d()}{\parsemathoperator{\operatorname{ker}}{#1}{#2}} % improvement of previous ker
\NewDocumentCommand{\imag}{g d()}{\parsemathoperator{\operatorname{im}}{#1}{#2}}
\NewDocumentCommand{\subgr}{}{\ensuremath{\leqslant}}
\NewDocumentCommand{\rsubgr}{}{\ensuremath{\geqslant}}
\NewDocumentCommand{\normsubgr}{}{\ensuremath{\trianglelefteqslant}}
\NewDocumentCommand{\rnormsubgr}{}{\ensuremath{\trianglerighteqslant}}
\NewDocumentCommand{\grcenter}{g d()}{\parsemathoperator{\operatorname{Z}}{#1}{#2}}
\NewDocumentCommand{\centralizer}{m g d()}{\parsemathoperator{\operatorname{C}_{#1}}{#2}{#3}}
\NewDocumentCommand{\normalizer}{m g d()}{\parsemathoperator{\operatorname{N}_{#1}}{#2}{#3}}
\NewDocumentCommand{\factorgr}{m m}{\ensuremath{\nicefrac{#1}{#2}}}
\NewDocumentCommand{\rfactorgr}{m m}{%
    \reflectbox{%
        \nicefrac{\reflectbox{\ensuremath{#1}}}%
        {\reflectbox{\ensuremath{#2}}}}%
}
\NewDocumentCommand{\Cl}{g d()}{\parsemathoperator{\operatorname{Cl}}{#1}{#2}}
\NewDocumentCommand{\Stab}{g d()}{\parsemathoperator{\operatorname{Stab}}{#1}{#2}}




%=============================== Theorems ===============================
\newtheoremstyle{plainbreak}
{}%                                  % Space above, empty = `usual value'
{}%                                  % Space below
{\rmfamily}%                         % Body font
{}%                                  % Indent amount (empty = no indent, \parindent = para indent)
{\bfseries}%                         % Thm head font
{}%                                  % Punctuation after thm head
{\newline}%                          % Space after thm head: \newline = linebreak
{}%                                  % Thm head spec

\theoremstyle{plainbreak}
% Name in code, name in text, counter dependency
\newtheorem{theorem}{Теорема}[section]
% Counter here is placed in between, which means that it is not dependent but the same counter
\newtheorem{proposition}[theorem]{Утверждение}
\newtheorem{corollary}{Следствие}[theorem]
\newtheorem{lemma}[theorem]{Лемма}
\newtheorem*{example}{Пример}
\newtheorem*{examples}{Примеры}
\newtheorem{problem}{Задача}[section]
\newtheorem{definition}{Определение}[section]
\newtheorem{denotation}[definition]{Обозначение}

\theoremstyle{remark}
\newtheorem*{remark}{Замечание}
\newtheorem*{solution}{Решение}


\newcommand*\@proofenvname{proof}
\newcommand*{\theoremlistshack}{%
    \leavevmode
    \ifx\@currenvir\@proofenvname
    \else
        \vspace{-\baselineskip}
    \fi
    \par%
    \everypar{\setbox\z@\lastbox\everypar{}}%
}

\makeatother
